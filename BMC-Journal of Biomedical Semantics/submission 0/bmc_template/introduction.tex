
\section*{Introduction}
Data and Information modeling in the healthcare domain have witnessed significant improvements in the last decade owing to advances in the development of state-of-the-art information and communication technologies (ICT) and formalization of storage and messaging standards. Subsequently, the scope of Healthcare Management Information Systems (HMIS), medical ontologies, and Clinical Decision Support Systems (CDSS) has broadened, beyond the operational capabilities of traditional rule based systems. One of the major reasons behind this limitation is due to the numerous heterogeneities in healthcare at data, knowledge, and process level. Thus, healthcare interoperability which aims to provide a solution to this problem, can be compartmentalized into data interoperability, process interoperability, and knowledge interoperability.
Data interoperability resolves the heterogeneity between data artificats, to enable, seamless and interpretable communcation among source and target organizations, while preserving the data's original intention during storage, communication, and usage (as defined by IEEE 610.12 \cite{geraci1991ieee}, Health Level Seven International HL7, and Healthcare Information and Management Systems Society HIMSS \cite{Himss_url2013}).
On the other hand, process interoperability regulates the communication among organizational processes to provide compatability between process artifacts within and seamless transformations across different organizations\cite{khan2013process}. Lastly, knowledge interoperability provides a sharing mechanism for reusing interpretable medical knowledge, acquired through expert intervention and other mechanisms, across decision support systems \cite{ali2017multi}.

In more tangible terms, healthcare interoperability at data, process, and knowledge level can be exemplified within the healthcare constraints experienced due to the emergence of Covid 19. The operational capabilities of the current healthcare service delivery infrastructure has gone under tremendous stress due to Covid 19. World over, large primary healthcare units have managed to create separate units for managing patients, suffering from extreme cases of the novel coronavirus. For secondary and tertiary care units, government involvement has become necessary to filter coronavirus patients and adhering to a national pandemic response policy. 
These complex circumstances have enhanced the need for sharing patient data and state-of-the-art medical knowledge in real-time, to provide the medical experts with a tool to make accurate and timely decisions.
Data interoperability can enable the front line medical workers to fetch, understand, and use patient data, especially comorbidities, across organizational and physical boundaries, without suffering from societal taboos that may prevent the patient from sharing their complete and accurate medical histories. Knowledge interoperability can improve the knowledge acquisition and sharing protocols to provide the medical experts such as epidieomologists and vaccinologist, with latest information on affected population trends, disease diagnosis, treatment, and followup procedures, and interpretable decisions leading to positive or negative outcomes. Process interoperability can help reduce and in some cases remove the operational redundancies between health centers. In this way, successive healthcare treatments can take benefit from earlier diagnosis, treatment, and followup procedures, thereby reducing the stress on healthcare experts and systems.

Standards such as HL7 and openEHR provide the foundations for storing and communicating medical data, through the use of well defined protocols such as HL7-FHIR, or the older HL7v2  HL7 v3. While Snomed-CT and LOINC provide a standard definition for clinical terminologies and lab tests, respectively. Similarly Medical Logic Module (MLM) provides a standardized way for expressing medical knowledge. However, the plethora of standards,n necessitates the creation of bridging standards, that can resolve the heterogeniety between medical standards. In this regard, CIMI has Cimi and Yosemite 

Uhp, previous paper



pharmacists and vaccinologists 
Under these complex circumstances, a technical solution for healthcare interoperability, has become all the more necessary. While the standardization efforts by CIMI, SNOMED-CT, LOINC, yosemite group 