
\section*{Results}
\label{results}









The validity of our proposed approach has been evaluated using several techniques, including comparison of the proposed syntactic and semantic matching process with fuzzy string matching, only semantic matching of attribute names using sentence embedding, and with multiple human annotated versions of the schema map. Using the 21,873 AA pairs, we created a two dimensional matrix, of AA identifiers, whereby each cell refers to a ``SchemaName\_TableName\_AttributeName" element. All nodes, for which the SchemaName part of the name is same, are then marked with ``-", and these are omitted from further processing. This various versions of this dataset (D) are available at \textbf{url}.

The results of fuzzy string comparison ($D_f$) on the attribute names of each pair, can be visualized using the heatmap shown in \textbf{figFuzzyWuzzy}. Similarly, the results of comparison using BERT based word embedding of the attribute names ($D_b$), is shown in \textbf{figBase}. Finally, the heatmap of the syntactic and semantic matching of the AA's using our proposed Algorithm \ref{knowledgeEvolAlgo} ($D_p$), is shown in \textbf{figSynAndSemMatching}. The values, for each cell range between ``0" and ``1", with higher values indicating a larger match between the elements, in each of these cases. The black areas of these heatmaps, indicate missing values, while white indicates a matching value of ``0". The few areas with red color, indicate the values above ``0.9", while blue areas indicate values above ``0.8". 

\textbf{Threshold Selection}
Overall, the number of similarities above the threshold values of ``0.8" are greatly increased from $D_f$, to $D_b$, and then to $D_p$ (proposed methodology). Changing the threshold value in $D_p$ to ``0.9", brings it relatively closer to $D_b$. This is due to our proposed similarity matching methodology, which gives an equal score (0.5) to the values obtained from both syntactic and semantic matches. Since the decision made at the syntactic evaluation is binary (0 or 1), thus for a similarity score of ``0.8", in $D_p$, ``0.5" represents the syntactic similarity, and ``0.3" semantic similarity. The semantic similarity value, obtained from this rescaling process, would in turn represent an original value of ``0.6" from $D_b$. Therefore, for an equal semantic representation, the value of semantic similarity after rescaling (between 0 and 0.5) should be ``0.4", and originally ``0.8". \textbf{figSynAndSemMatching0.9} shows the heatmap, of similarity matrix, obtained from $D_p$ with the threshold values as ``0.9".

Generally, a comparison of these heatmaps, indicate an increase in the recall rate of the methodology. This is due to the fact, that our proposed methodology is able to match a larger number of AAs, providing 50\% of the score to the syntactic match, thereby increasing the count of cells without a missing value. This approach is also able to reduce the impact of semantic matching from names, which might contain repeated words, such as ``patient" but in reality refer to disparate concepts, such as ``PatientName" and "PatientDateOfBirth". Another important aspect of our matching process is the inclusion of semantic concepts and suffixes, as a sentence before its conversion to an embedded vector. As discussed earlier, this process is used to reduce the impact of syntactic matches alone, which would give a higher score to the attributes ``PatientName" and ``PatientGender", which refer to different concepts. The inclusion of concepts behind ``Name" and ``Gender" part of these names, allow for an improved comparison. 


In order to evaluate the accuracy and precision rates of these methodologies, we then compared the results of our computed methods $D_f$, $D_b$, and $D_p$, with the average scored datasets by the human annotators. A visualization of the resultant Cohen's Kappa score for the 261 participating attributes, is shown in \textbf{figComputedComparisonKappaScore}. The bar chart shown in this visualization, indicates, an overall greater positive agreement between our proposed methodolody $D_p$, and the average scores of the annotators. The same is shown by the \textbf{orange} colored peaks in this dataset, which in all cases is either equivalent to the other approaches ($D_f$ and $D_b$), or better than them.


\begin{table}[]
	\begin{tabular}{|l|l|l|l|l|l|}
		\hline
		Method     & Total Matches & Marked as Equal & Marked as Related & Marked as Unrelated & Not Marked \\ \hline
		Annotator1 & 40698         & 238             & 109               & 40351               & 0          \\ \hline
		Annotator2 & 40698         & 241             & 116               & 40341               & 0          \\ \hline
		Annotator3 & 40698         & 260             & 2103              & 38182               & 153        \\ \hline
		Annotator4 & 40698         & 225             & 62                & 40400               & 11         \\ \hline
	\end{tabular}
\end{table}




