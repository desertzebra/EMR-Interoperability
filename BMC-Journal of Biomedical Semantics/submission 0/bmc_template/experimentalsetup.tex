
\section*{Experimental Setup}
\label{experimentalSetup}
In our earlier work \cite{Satti2020} $s_1$, $s_2$, and $s_5$ were used to generate over 115 million patient records, which are converted into a semi-structured form and stored in Hadoop Distributed File System (HDFS). We extended the same setup to create an additional 100,000 records, for 1000 patients with 3 medical fragments for $s_1$, $s_2$, and $s_4$, and 97 randomly selected and generated medical fragments amongst $s_1$, $s_2$, $s_3$, $s_4$ and $s_5$. These fragments, follow various design elements, supporting a variety of valid relational storage architectures. Such as, $s_1$, $s_2$ and $s_4$ are represented by creating a separate medical fragment for each participating table, $s_3$ utilizes its medical fragment to generate a linked record (from a linked object graph), where by the attributes can refer to other objects besides the elements of $t$, mimicking the application of explicit foreign keys, and $s_5$ is a flat table structure. The code to generate this data set is available at ``uhp\_map\_generation"\footnote{https://github.com/desertzebra/UHP\_v4/tree/main/uhpr\_storage}. This application produces three custom formatted files, containing an index for patients, an index for their medical fragments, and the medical fragment, corresponding to the EMR data. Using the medical fragments file, we then generate the AA\footnote{https://github.com/desertzebra/UHP\_v4/tree/main/uhp\_map\_generation}. The resulting set of AA are temporarily stored in a ``json" file, which is then read by the same application to partially generate the schema maps. This process, is used to create 13,336  distinct pairs of attributes, across $s$. Each pair also contains the ``relationshipList", which stores either the positive result of a case insensitive match between the attributes, in which case no further processing is performed, or the results of fuzzywuzzy matching between the attribute names. 
The ``json" file thus produced, is then used by a python script to perform the syntactic and semantic matching on the attributes. In this script we used the pre-trained \textbf{Bert base model with nli mean tokens}. 

The rationale behind switching the applications at various stages, is to cache the results and create checkpoints for restarting any failed stages, easily. Additionally, since python provides better support for easy generation of embedding vectors, it was thus preferred over the Java based implementation, which is otherwise very beneficial for other tools. These applications were executed on a workstation running Ubuntu 20.04.2 LTS on top of AMD Ryzen 3 2200G, and 32GB ram.



