
\section*{Experimental Setup}
\label{experimentalSetup}
In our earlier work \cite{Satti2020} $s_1$, $s_2$, and $s_5$ were used to generate over 115 million patient records, which are converted into a semi-structured form and stored in Hadoop Distributed File System (HDFS). We extended the same setup to create an additional 100,000 records, for 1000 patients with 3 medical fragments for $s_1$, $s_2$, and $s_4$, and 97 randomly selected and generated medical fragments amongst $s_1$, $s_2$, $s_3$, $s_4$ and $s_5$. These fragments, follow various design elements, supporting a variety of valid relational storage architectures. Such as, $s_1$, $s_2$ and $s_4$ are represented by creating a separate medical fragment for each participating table, $s_3$ utilizes its medical fragment to generate a linked record (from a linked object graph), where by the attributes can refer to other objects besides the elements of $t$, mimicking the application of explicit foreign keys, and $s_5$ is a flat table structure. The code to generate this data set is available at ``uhp\_map\_generation"\footnote{https://github.com/desertzebra/UHP\_v4/tree/main/uhpr\_storage}. This application produces three custom formatted files, containing an index for patients, an index for their medical fragments, and the medical fragment, corresponding to the EMR data. Using the medical fragments file, we then generate the semantically enriched attribute \footnote{https://github.com/desertzebra/UHP\_v4/tree/main/uhp\_map\_generation}, which contains the suffixes and their concepts corresponding to each EMR data attribute. The resulting set of enriched attributes are temporarily stored in a ``json" file, which is then read by the same application to partially generate the schema maps. This process, is used to create 21,873  distinct pairs of attributes, across $s$. Each pair also contains the ``relationshipList", which stores the results of fuzzy string matching\cite{FuzzyWuzzy} \footnote{Java Library: https://github.com/xdrop/fuzzywuzzy} between the attribute names. 
The ``json" file thus produced, is then used by a python script to generate the semantically enriched sentences and their embedded vectors using WordNet, and 10 pre-trained BERT NLI models \cite{reimers-2019-sentence-bert}. The BERT models include 'bert-base-nli-stsb-mean-tokens', 'bert-large-nli-stsb-mean-tokens', 'roberta-base-nli-stsb-mean-tokens', 'roberta-large-nli-stsb-mean-tokens', 'distilbert-base-nli-stsb-mean-tokens', 'bert-base-nli-mean-tokens', 'bert-large-nli-mean-tokens', 'roberta-base-nli-mean-tokens', 'roberta-large-nli-mean-tokens', and 'distilbert-base-nli-mean-tokens'. 
The embedding vectors are then compared using cosine similarity, which produces a score between -1 and 1.
The rationale behind switching the applications at various stages, is to cache the results and create checkpoints for restarting any failed stages, easily. Additionally, since python provides better support for easy generation of embedding vectors, it was thus preferred over the Java based implementation, which is otherwise very beneficial for other tools. These applications were executed on a workstation running Ubuntu 20.04.2 LTS on top of AMD Ryzen 3 2200G, and 32GB ram.

In order to compare our computed models with ground truth, and to identify the best thresholds for classifying each instance as ``equal", ``related", or ``unrelated" 4 human annotators were utilized, to anonymously, score the similarity of each pair of attribute names. In order to support this process, we first repurposed one of our generated data matrix, by marking all attribute pairs belonging to the same schema with the symbol ``-". Following this, the annotators, marked each cell corresponding to a pair of attributes, by determining the similarity in terms of dissimilar as ``0", exactly similar as ``1", row attribute as child of column attribute as ``<", row attribute as a parent of the column attribute as ``>", and finally, unknown as ``~". The data sheets generated after this extensive human effort, have been made available for other researchers\footnote{https://github.com/desertzebra/EMR-Interoperability/tree/master/Implemenation/Data/Annotated}. These sheets, additionally contain some missing values, which were left out by the annotators, but in order to maintain their originality, these values were not filled; instead during our evaluation for these datasets, the missing values were considered as having the score ``0". Using \textbf{Cohen's Kappa score (d)}, we evaluated the inter-rater agreement of these annotations, which have been visualized in \textbf{figKappaScore}. The 7 permutations, amongst the 4 annotators, generate 261 kappa scores, where each individual score corresponds to the correlation of agreement between two annotators. Further, each score identifies the agreement in terms of scoring a set of paired attributes, while keeping the row attribute constant, and moving along the 261 column attributes. Finally we produced a consolidated dataset, using mode scores of all annotators, for each attribute pair. This dataset is then split into development and testing partitions with a ratio of 70:30. The development partition is used for threshold selection, based on the best MCC score for identifying class ``equal", followed by best scores for class ``related" and finally best of class "unrelated". The optimal threshold thus achieved is used to classify the instances of the test dataset, which is finally evaluated on its MCC and F-1 score.

