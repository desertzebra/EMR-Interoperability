
\section*{Methodology}
\label{methodology}
Healthcare interoperability, with a focus on non-standard compliant medical schema, is dependent on the generation and validation of schema maps, as discussed above. To this end, the creation of a cohesive workflow is of utmost importance. In our earlier work \cite{Satti2020} we used maximum sequence identification and matching using suffix trees for syntactic matching of two distinctly sources data schemas. This was followed by semantic concept enrichment and subsequently concept matching, for creating rules in the form of schema maps. The simplified mapping functions, thereby created, provided a simple methodology for converting semi-structured medical data, into a non-persisted, interpretable, model form.
In our current methodology we have utilized state-of-the-art machine learning techniques for converting medical schemas into semi-structured form, which is then used to create embedding vectors. These vectors are then used for modeling and creating interpretable rules for applying semantic reconciliation in the form of schema matching and/or transformation. These rules are managed by the RDR which allows fast inference using the relevant knowledge sub-tree, and can well manage knowledge evolution to scale out (by incorporating new schema) and up (by incorporating changes and additions to existing schema)
As a possible realization of this aim, and from a practical point of view, our methodology is presented in the following subsections.

\subsection*{Schema acquisition}
\label{schema_acq}
In the first step of our semantic reconciliation methodology, we collect medical schema from five distinct EMR storage systems ($S$). These include patient reports from OpenEMR ($s_1$), 100,000 patient records from EMRBOTS ($s_2$) \cite{kartoun2016methodology}, custom database design by Pan et. al($s_3$) for supporting regional clinics and health care centers in China \cite{pan2016design}, clinical knowledge discovery tool MedTAKMI-CDI ($s_4$) \cite{inokuchi2007medtakmi}, and our custom implementation ($s_5$). These five schema follow relational database design, with logical entities, such as demographics, diagnosis, medicine or others, placed into tables which can be further linked to one or more tables. Each of these schema, fulfills the need of their respective information processing applications, however the lack of interoperability is very much visible here. Here, heterogeneity is caused by various factors, including the lack of standard terminologies and different normalization level.
While SNOMED-CT provides a mechanism for identifying the standard codes for clinical terms and LOINC can be used for laboratory related terms, most attribute names are created based on the gut feeling of the database designer. While this behavior is not wrong, it does create syntactic differences in the participating schema. Consider the terms ``name" and ``patientName", which refer to the same attribute of the patient entity, however there is no standard way to represent it, with both being correct. As a result, $s_1$ and $s_3$ use the former representation, while $s_2$ and $s_5$ use the latter. 
Differences in normalization cause semantic differences, due to which some data could be available in one schema but absent in others, such as OpenEMR demographics identifying the patient's residential location using specific attributes like ``Address", ``City", ``State", ``Postal Code", ``Country", and others. Similarly, ``EncounterDate" from $s_5$ is semantically similar to ``BeginDate" of ``openemr\_MedicalProblems" table in $s_1$, ``AdmissionStartDate" of ``LabsCorePopulatedTable" in $s_2$, 
``time" in ``Diagnosis" table of $s_3$, and  ``dateOfAdmission" in ``Diagnosis" and ``CareHistory" tables of $s_4$.
Finally, $s_1$ and $s_3$ have separate tables containing the medicinal prescription details, however the same details are unavailable in $s_2$, $s_4$, $s_5$. Once again, this is not an incorrect behavior since this information, might not be a part of the context or the requirements for these EMR storage systems.
In fact, the change in context of the EMR storage system from the initial time of development to a later stage of collaborative processing systems, is the main cause of heterogeneity.
In order to provide an interoperable solution, it is therefore necessary to provide syntactic and semantic mappings, while taking the data instances (or tuples) into account. Data instances are used to enrich the semantic context of an attribute ($a_i$) by establishing its probable data type and possible value set. 
In our earlier work \cite{Satti2020} $s_1$, $s_2$, and $s_5$ were used to generate over 115 million patient records, which are converted into a semi-structured form and stored in Hadoop Distributed File System (HDFS). We extended the same setup to \textbf{further include}

In essence, the set of attribute ($A$) holds all unique attributes of all participating schema.



\subsection*{Schema processing}
\label{schema_prepro}
The medical schemas and their instances are converted into an interpretable form with each attribute containining:

schema name, attribute name, min value, max value, type, semantic types[]



Methodology goes here ....