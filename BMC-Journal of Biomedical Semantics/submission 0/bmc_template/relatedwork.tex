
\section*{Related Work}
\label{relatedwork}
Althubait et al. \cite{althubaiti2020combining} proposed an ontology expansion methodology that identifies and extracts new class from text articles using word embedding and machine learning techniques. The authors identified the similarity of tokens and phrases of the text articles with the exiting classes of the ontology. The target ontology is expanded with classes from text articles having greater similarity with that of already added classes. A similar word embedding technique was also used by Nozaki et al. \cite{nozaki2019semantic}, where the authors used instance based schema matching technique to identify the semantic similarity between two instances. The results of the study showed the possibility of detecting similar string attributes of different schemas. Yousfi et al. \cite{yousfi2020xmatcher} also utilized semantic base techniques and proposed xMatcher XML schemas matching approach. xMatcher transforms schemas into a set of words, followed by measuring words context, and relatedness score using WordNet. The terms from different schemas having similarities greater or equal to 0.8 are considered similar. 
Bylygin et al. \cite{bulygin2018combining} devised an ontology and schema matching approach by combining lexical and semantic similarity with machine learning approaches.  The authors used lexical and semantic measures as features and trained various machine learning algorithms including Naive Bayes, logistic regression, and gradient boosted tree. The result achieved showed that the combination of algorithms outperformed the single modal.

Martono et al. \cite{martono2017review} provided overview of previous studies related to linguistic approaches used for schema matching. Linguistic methods focused on finding strings and evaluate there similarity in different schemas. The string are normally normalized before to align both the strings before similarity comparison. The normalized strings are categories based on the information relatedness and element with similar category are compared using various similarity measure including  Jaro-distance, Lavenstein (edit-distance), and many more. Alwan et al. \cite{alwan2017survey} have summarized the techniques used in the literature for schemas and instances based schema matching. The information used for schema matching is categories into  schema information, instance and auxiliary information. Most of the searchers have used syntactic techniques (including n-gram, and regular expression), semantic techniques (including Latent Semantic Analysis, WordNet/Thesaurus and Google Similarity) for schema level and instance level matching to achieve the final goal of data/information interoperability.  Kersloot et al. \cite{kersloot2020natural} performed a comprehensive systematic review to evaluate natural language processing (NLP) algorithms used for clinical text mapping onto ontological concepts. The findings of the studies were evaluated with respect to five categories; use of NLP algorithms, data used, validation and evaluation performed, result presentation, and generalization of results. The authors revealed that over one-fourth of the NLP algorithms used were not evaluated and have no validation. The systems that claimed generalization, were self evaluated and having no external validation.  

Xu et al. \cite{xu2003discovering} presented a framework for discovering indirect links besides direct links among schema elements. The indirect matches were detected for relations such as union, composition, decomposition, selection, and boolean. The indirect links are useful to handle concepts merge, split, generalization, and specialization. The matching techniques utilized in the study considered terminological relationships (word synonym and hypernym), structural characteristics, data-value characteristics, and expected data values. The experimental results revealed framework effectiveness by achieving more than 90\% precision and recall for direct and indirect link matching. 

A comprehensive survey from 176 experts including physicians and nurses was conducted by Moll et al. \cite{moll2020oncology} to check their perspectives regarding patient accessible electronic health record (PAEHR). The authors discovered that the PAEHR positively effect after six years of operations despite negative expectations. The primary concerns revealed was logger meeting time, change in documentation practices, and increasing varies of patients regarding their health conditions. However, attitude of both healthcare service providers and patients are changing positively with respect to PAEHR and its benefits. 