
\section*{Related Work}
Althubait et al. \cite{althubaiti2020combining} proposed an ontology expansion methodology that identifies and extracts new class from text articles using word embedding and machine learning techniques. The authors identified the similarity of tokens and phrases of the text articles with the exiting classes of the ontology. The target ontology is expanded with classes from text articles having greater similarity with that of already added classes. A similar word embedding technique was also used by Nozaki et al. \cite{nozaki2019semantic}, where the authors used instance based schema matching technique to identify the semantic similarity between two instances. The results of the study showed the possibility of detecting similar string attributes of different schemas. Yousfi et al. \cite{yousfi2020xmatcher} also utilized semantic base techniques and proposed xMatcher XML schemas matching approach. xMatcher transforms schemas into a set of words, followed by measuring words context, and relatedness score using WordNet. The terms from different schemas having similarities greater or equal to 0.8 are considered similar.  

Kersloot et al. \cite{kersloot2020natural} performed a comprehensive systematic review to evaluate natural language processing (NLP) algorithms used for clinical text mapping onto ontological concepts. The findings of the studies were evaluated with respect to five categories; use of NLP algorithms, data used, validation and evaluation performed, result presentation, and generalization of results. The authors revealed that over one-fourth of the NLP algorithms used were not evaluated and have no validation. The systems that claimed generalization, were self evaluated and having no external validation.  
